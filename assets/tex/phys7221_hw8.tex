\documentclass{jhwhw}

\author{PHYS 7221}
\title{Homework \#8}
\date{Due: 8 November 2022}

\begin{document}

\problem{[20 pts] Moving pendulum}

The point of suspension of a simple pendulum of length $\ell$ and mass $m$ is constrained to move on a parabola described by $z = \kappa x^{2}$ with constant $\kappa$ and located in the vertical plane ($y = 0$). Derive the Hamiltonian governing the motion of the pendulum from the Lagrangian. Obtain Hamilton's equations.

\problem{[20 pts] Multiple integrals of motion}

Consider a system with Hamiltonian $H = q_{1} p_{1} - q_{2} p_{2} - a q_{1}^{2} + b q_{2}^{2}$ where $a$ and $b$ are real numbers. Show that the functions $f_{1} = (p_{2} - b q_{2})/ q_{1}$ and $f_{2} = q_{1}q_{2}$ are integrals of motion. Verify that their Poisson bracket is another constant of the motion (Poisson's theorem).

\problem{[20 pts] Canonical transformations}

Show that the following coordinate transformation is a canonical transformation
\begin{equation}
  \label{eq:1}
  Q = \arctan\left(\frac{\alpha q}{p} \right), \quad P = \frac{\alpha q^{2}}2\left( 1 + \frac{p^{2}}{\alpha^{2} q^{2}}\right).
\end{equation}
where $\alpha$ is a real number.

\problem{[20 pts] Canonical transformations in action}

Show that the following coordinate transformation of a system with two degrees of freedom (i.e., a four-dimensional phase space) is a canonical transformation
\begin{align}
  \label{eq:2}
  x & = \frac1{\alpha} ( \sqrt{2 P_{1}} \sin Q_{1} + P_{2} ), & y = \frac1{\alpha}( \sqrt{2 P_{1}} \sin Q_{1} + Q_{2} ) \\
  p_{x} & = \frac \alpha 2( \sqrt{2 P_{1}} \sin Q_{1} - Q_{2} ), & p_{y} = - \frac{\alpha}2 ( \sqrt{2 P_{1}} \sin Q_{1} - P_{2} )
\end{align}
Apply this canonical transformation to a particle of charge $e$ moving in a the $xy$-plane orthogonal to the magnetic field $\mathbf B = B \hat{\mathbf z}$, where $\alpha^2 = \frac{e B}{c}$ and solve the equations of motion.

\emph{Hint:} You know the form of the Hamiltonian in terms of $x$, $y$, $p_{x}$, and $p_{y}$ from in class and in the previous homework.
What you need to do is to change to the new variables $Q_{1}$, $Q_{2}$, $P_{1}$, and $P_{2}$.
You will find that the system can be solved quite easily using these variables (this is an example of how useful canonical transformations are for solving the equations of motion).
Once you find the solution in terms of $Q_{1}$, $Q_{2}$, $P_{1}$, and $P_{2}$, undo the change to write it in terms of the original coordinates $x$, $y$, $p_{x}$, and $p_{y}$.

\end{document}
