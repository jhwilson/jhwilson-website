\documentclass{jhwhw}

\author{PHYS 7221}
\title{Homework \#1}
\date{Due: 30 August 2022}

\begin{document}

\problem{[12pts] Some basic math of classical mechanics}

Classical mechanics requires knowledge of linear algebra, differential equations, and approximations.
The goal of this question is to get you acquainted and in the right mathematical mindset for this course
\begin{enumerate}
  \item {[4pts]} Linear Algebra: Consider the matrix
        \begin{equation}
          \label{eq:5}
          M =
          \begin{pmatrix}
            a & b & 0 & b \\
            b & a & b & 0 \\
            0 & b & a & b \\
            b & 0 & b & a \\
          \end{pmatrix},
        \end{equation}
      find the eigenvalues, their degeneracies, and the four eigenvectors. Show that the eigenvectors are orthogonal (i.e., the dot product $\sum_{i} v_{i} w_{i} = 0$ for different eigenvectors $v$ and $w$).
  \item {[4pts]} Differential equations: a mass on a spring obeys the equation
        \begin{equation}
          \label{eq:6}
          m \ddot x(t) = - k x(t), \quad x(0) = x_{0}, \quad \dot x(0) = v_{0}.
        \end{equation}
        Derive the full solution for $x(t)$. What is the frequency spring in radians per second? What is the period?
  \item {[4pts]} Taylor expansions: We will often use approximations and one classic example of this is the Taylor expansion $f(y) = \sum_{n=0}^{\infty} \frac{f^{(n)}(0)}{n!} y^{n}$. In the following expressions, write the first four terms in the Taylor expansion in the variable $x$ assuming $a$ is constant (up to the $x^{3}$ term): (1) $1/(x-a)$, (2) $\sin(a x)/x$, and (3) $\exp(x \cos(a x))$.
\end{enumerate}


\problem{[12pts] Functional derivatives}

In class we discussed functional derivatives and extremizing the action.
In this problem, we will build up this idea from basic calculus
\begin{enumerate}
  \item {[4pts]} Single variable: Find the minimum of the function $f(x) = a x^{2}+ b x + c$ assuming $a>0$.
  \item {[4pts]} Multi-variable: Find the extremal value for the function
        \begin{equation}
          \label{eq:1}
          f(x,y,z) = a(x^{2}+y^{2}+z^{2}) + 2b(xy + yz + xz) + J x + 2J y - J z.
        \end{equation}
  \item {[4pts]} Even more variables: Determine what equations are needed to solve to find the extremum of the following expression with variables $x_{1}, x_{2}, \ldots, x_{N}$ (you do not have to solve them)
        \begin{equation}
          \label{eq:2}
          S(x_{1}, x_{2}, \ldots, x_{N}) = \sum_{i=2}^{N} a (x_{i}- x_{i-1})^{2} - \sum_{i=1}^{N} b_{i} x_{i}^{2}.
        \end{equation}
  \item {[4pts]} Continuous variables: In Eq.~\eqref{eq:2} let $a = \frac{m}{2 \Delta t}$ and $b_{i} = \frac12 k_{i} \Delta t$, and $\Delta t = T/N$. Show that in the limit of $N\rightarrow \infty$ that Eq.~\eqref{eq:2} becomes an integral,
        \begin{equation}
          \label{eq:3}
          S[x] = \int_{0}^{T} L[x(t), \dot x(t), t] dt.
        \end{equation}
  Derive what $L[x(t), \dot x(t), t]$ is and show that the Lagrange equations of motion you derive from it match the equations from part (c) above when $N\rightarrow \infty$.
\end{enumerate}

\problem{[30pts] Lagrangians, constraints, and equations of motion}

Write the constraints, find generalized coordinates, write the Lagrangian, and find the equations of motion for the following physical systems
\begin{enumerate}
  \item {[10pts]} Double pendulum (see figure below).
  \item {[10pts]} Pendulum on a slider (see figure below).
  \item {[10pts]} Pendulum with a pivot point forced to rotate with angular velocity $\omega$ (see figure below).
\end{enumerate}
\includegraphics[width=\textwidth]{hw1_pendulums.pdf}

\end{document}
