\documentclass{jhwhw}

\author{PHYS 7221}
\title{Homework \#9}
\date{Due: 15 November 2022}

\begin{document}

\problem{[20pts] Runge-Lenz vector redux}

Consider a particle in \textbf{three dimensions} moving under the action of the Kepler potential $V(r)=-\frac kr$.
Show that the Runge-Lenz vector
\begin{equation}
  \label{eq:1}
  \mathbf A = \mathbf p \times \mathbf L - \frac{m k \mathbf r}{r},
\end{equation}
is an integral of motion (i.e., $\{ \mathbf A, H\} = 0$).

\problem{[20 pts] Generators of canonical transformations}

Consider a mechanical system with one degree of freedom.
Obtain the generating function $F_{3}(p,Q)$ that generates the same canonical transformation as $F_{2}(q,P)= q^{2} e^{P}$.

\problem{[20 pts] Building canonical transformations}

Let
\begin{equation}
  \label{eq:2}
  Q_{1} = q_{1}^{2}, \quad Q_{2} = q_{1} + q_{2}, \quad P_{i} = P_{i}(q_{j},p_{j}), \quad i=1,2
\end{equation}
be a canonical transformation of a system with two degrees of freedom.
\begin{enumerate}
  \item Complete the transformation by finding \emph{the most general} expression for the $P_{i}$'s.
  \item Find \emph{a particular} choice of the $P_{i}$'s that will reduce the Hamiltonian
        \begin{equation}
          \label{eq:3}
          H(q_{i},p_{i}) = \left(\frac{p_{1}-p_{2}}{2q_{1}}\right)^{2} + p_{2} + (q_{1} + q_{2})^{2}
        \end{equation}
        to
        \begin{equation}
          \label{eq:4}
         H_{\mathrm{new}}(Q_{i},P_{i}) = P_{1}^{2} + P_{2}.
        \end{equation}
  \item Use $H_{\mathrm{new}}(Q_{i},P_{i})$ to solve for $q_i(t)$.
\end{enumerate}

\end{document}
