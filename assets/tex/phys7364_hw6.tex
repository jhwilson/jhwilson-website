\documentclass{jhwhw}

\usepackage{amsfonts}
\usepackage{listings}

\DeclareMathOperator{\Rer}{Re}
\DeclareMathOperator{\Imr}{Im}

\author{}
\title{PHYS 7364 -- Homework \#6}
\date{4 Apr 2022}

\begin{document}

\problem{Second quantization}

In this question $c_{i}^{\dagger}$ and $c_{i}$ are fermion creation and annihilation operators and the states are fermion states.
Use the convention $\ket{1111100\cdots} = c_{5}^{\dagger} c_{4}^{\dagger}c_{3}^{\dagger} c_{2}^{\dagger} c_{1}^{\dagger} \ket{000 \cdots}$.
\begin{enumerate}
  \item Use the anticommutation relations for fermions to ``normal-order'' $c_{3}^{\dagger} c_{6} c_{4} c_{6}^{\dagger} c_{3}$ (``normal-order'' means commuting all annihilation operators to the right, so for instance $ c_{2}c_{1 }c_1^{\dagger}$ normal ordered would be $-c_{1}^{\dagger} c_{1}c_{2} + c_{2}$).
  \item Evaluate $c_{3}^{\dagger} c_{6} c_{4} c_{6}^{\dagger} c_{3}\ket{111111000\cdots}$ and $c_{3}^{\dagger} c_{6} c_{4} c_{6}^{\dagger} c_{3}\ket{111110000\cdots}$.
  \item Write $\ket{1101100100\cdots}$ in terms of excitations about the ``filled Fermi sea'' $\ket{\Omega} = \ket{1111100000\cdots}$.  Interpret your answer in terms of electron and hole excitations.
  \item Find $\braket{\psi | \hat{N} | \psi}$, where $\ket{\psi} = A\ket{100000} + B \ket{111000}$, $\hat{N} = \sum_{i} c_{i}^{\dagger} c_{i}$.
\end{enumerate}

\problem{Bogoliubov transformations}

Consider two fermions $a_{1}$ and $a_{2}$
\begin{enumerate}
  \item Show that the Bogoliubov transformation
        \begin{equation}
          \label{eq:1}
          \begin{aligned}
            c_{1} & = u a_{1} + v a_{2}^{\dagger}, \\
            c_{2}^{\dagger} & = - v a_{1} + u a_{2}^{\dagger}.
          \end{aligned}
        \end{equation}
        where $u$ and $v$ are real, preserves the canonical anticommutation relations if $u^{2} + v^{2} = 1$.
  \item Use this result to show that the Hamiltonian
        \begin{equation}
          \label{eq:2}
          H = \epsilon (a_{1}^{\dagger} a_{1} - a_{2} a_{2}^{\dagger}) + \Delta (a_{1}^{\dagger} a_{2}^{\dagger} + \mathrm{h.c.}),
        \end{equation}
        can be diagonalized in the form
        \begin{equation}
          \label{eq:3}
          H = \sqrt{\epsilon^{2} + \Delta^{2}}(c_{1}^{\dagger} c_{1} + c_{2}^{\dag} c_{2} - 1).
        \end{equation}
  \item What is the ground-state energy of this Hamiltonian?
  \item Write out the ground-state wavefunction in terms of the original operators $a_{1}^{\dagger}$ and $a_{2}^{\dagger}$ and their corresponding vacuum $\ket{0}$ (i.e., $a_{1,2} \ket{0} = 0$).
\end{enumerate}

\problem{Self-consistency in BCS superconductivity}

In class we derived the following Hamiltonian (which was valid for $\mathbf k$ near the Fermi surface)
\begin{equation}
  \label{eq:4}
  H = \sum_{\mathbf k,\sigma} (\epsilon_{\mathbf k} - \mu)c_{\mathbf k,\sigma}^{\dagger} c_{\mathbf k,\sigma} - \frac{g}{V} \sum_{\mathbf k, \mathbf k'} c_{\mathbf k, \uparrow}^{\dagger} c_{-\mathbf k,\downarrow}^{\dagger} c_{-\mathbf k', \downarrow} c_{\mathbf k',\uparrow}.
\end{equation}


\begin{enumerate}
  \item Repeat the mean-field ansatz $\Delta = - \frac{g}{V} \braket{\sum_{\mathbf k}c_{-\mathbf k', \downarrow} c_{\mathbf k',\uparrow} }$ to obtain the mean-field Hamiltonian
        \begin{equation}
          \label{eq:6}
          H_{\mathrm{MFT}} = \sum_{\mathbf k,\sigma} (\epsilon_{\mathbf k} - \mu)c_{\mathbf k,\sigma}^{\dagger} c_{\mathbf k,\sigma}  + \sum_{\mathbf k} [\Delta^{*} c_{-\mathbf k, \downarrow} c_{\mathbf k,\uparrow} + \mathrm{h.c.}] + \frac{V}{g} |\Delta|^{2}
        \end{equation}
  \item With what we learned in Problem 2, make a Bogoliubov transformation to put this Hamiltonian into the form
        \begin{equation}
          \label{eq:7}
          H_{\mathrm{MFT}} = \sum_{\mathbf k} E_{\mathbf k} (a_{\mathbf k,\sigma}^{\dagger} a_{\mathbf k, \sigma} - 1/2) + \frac{V}{g} |\Delta|^{2}.
        \end{equation}
\item What is the ground-state wave-function for this system (write in terms of the electron vacuum $\ket{0}$ and $c_{\mathbf k,\sigma}$ operators)? (Hint: Given the state $\ket{0}$ annihilated by $c$ operators, the state $a_{-\mathbf k,\uparrow} a_{\mathbf k,\downarrow} \ket{0}$ is annihilated by $a_{-\mathbf k, \uparrow}$ and $a_{\mathbf k, \downarrow}$.)
  \item Call the wave function from the previous part $\ket{\psi_{\mathrm{BCS}}}$. Show that the self-consistent equation derived from $\Delta = - \frac{g}{V} \braket{\sum_{\mathbf k}c_{-\mathbf k', \downarrow} c_{\mathbf k',\uparrow} }$ is
        \begin{equation}
          \label{eq:8}
           \Delta = g \int_{|\epsilon_{\mathbf k} - \mu|<\omega_{D}} \frac{d^{3}k}{(2\pi)^{3}} \frac{\Delta}{2\sqrt{(\epsilon_{\mathbf k} - \mu)^{2} + \Delta^{2}}}.
        \end{equation}
  \item Finally, find a nonzero approximate solution to (d) in terms of terms of the $g$, $\omega_{D}$, and the density of states at the Fermi level $\rho_{0}$. (Approximations are needed, if you get stuck, look up in a book that covers superconductivity).
\end{enumerate}

\problem{Braiding Majoranas}

Consider the following Hamiltonian for 4 Majorana fermions
\begin{equation}
  \label{eq:9}
  H = i \sum_{i=1}^{3} \Delta_{i} \gamma_{0} \gamma_{i}.
\end{equation}
This can be made by taking three wires in the following geometry and tuning the superconducting gap between neighboring pairs.

\begin{center}
\includegraphics[width=3cm]{majorana-wires.png}
\end{center}

In this problem, we are using $\{\gamma_{i}, \gamma_{j}\} = 2 \delta_{ij}$.

\begin{enumerate}
  \item Put the Hamiltonian in block-diagonal form with $\tilde \gamma_{\mu} = \sum_{\nu} O_{\mu\nu} \gamma_{\nu}$ such that
        \begin{equation}
          \label{eq:11}
          H = \frac{i}2 \tilde \gamma^{T}
          \begin{pmatrix}
            0 & \epsilon_{1} & 0 & 0 \\
            -\epsilon_{1} & 0 & 0 & 0 \\
            0 & 0 & 0 & \epsilon_{2}  \\
            0 & 0 & -\epsilon_{2} & 0
          \end{pmatrix} \tilde \gamma
        \end{equation}
        What are $\epsilon_{1}$ and $\epsilon_{2}$? (ensure $\tilde \gamma$ are properly normalized)
  \item Define two fermions $c_{1} = \frac12 (\gamma_1 - i \gamma_{2})$ and $c_{2} = \frac12(\gamma_{0} - i \gamma_{3})$, and define a basis for the Hilbert space as $\ket{11} = c_{2}^{\dagger} c_{1}^{\dagger} \ket{0}$, $\ket{10} = c_{1}^{\dagger}\ket{0}$, $\ket{01} = c_{2}^{\dagger} \ket{0}$, and $c_{1} \ket{0} = 0 = c_{2} \ket{0}$. What is the Hamiltonian $H$ in this basis? \emph{Hint}: It will be in the form:
        \begin{equation}
          \label{eq:12}
          H =
          \begin{bmatrix}
            H_{\mathrm{even}} & 0, \\
            0 & H_{\mathrm{odd}}.
          \end{bmatrix}
        \end{equation}
        where the rows are given by $\ket{00}$, $\ket{11}$, $\ket{01}$, and $\ket{10}$ with $H_{\mathrm{even}}$ and $H_{\mathrm{odd}}$ two-by-two matrices.
  \item Note that the parity operator $P = \gamma_{0} \gamma_{1} \gamma_{2} \gamma_{3}$ has eigenvalue $+1$ for $\ket{00}$ and $\ket{11}$ and $-1$ for $\ket{01}$ and $\ket{10}$. When $\Delta_{1} = 0 = \Delta_{2}$ what is the ground state manifold of $H$? Show that when we restrict to the ground state manifold that $P' = i \gamma_{1} \gamma_{2}$  acts the same as $P$.
  \item For $H_{\mathrm{even}}$ find the ground states when (1) $\Delta_{1,2} = 0$, (2) $\Delta_{2,3} = 0$, and (3) $\Delta_{1,3} = 0$. Compute the Berry phase for the path (1) $\rightarrow$ (2) $\rightarrow$ (3) $\rightarrow$ (1).
  \item Repeat (d) for $H_{\mathrm{odd}}$.
  \item Using the ground states and operator in (c), create a unitary $U = e^{i \phi P'}$ that changes each state by the Berry phase computed in (d,e).
  \item Compute $U\gamma_{1}U^{\dagger}$ and $U \gamma_{2} U^{\dagger}$ to show that these Majorana fermions were exchanged -- we have braided two Majoranas.
\end{enumerate}



\end{document}
