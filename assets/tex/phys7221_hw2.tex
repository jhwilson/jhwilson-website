\documentclass{jhwhw}

\author{PHYS 7221}
\title{Homework \#2}
\date{Due: 13 September 2022}

\begin{document}

\problem{[22 pts] Spring attached to a pivot point}

A spring of rest length $\ell$ and spring constant $k$ is connected at one end to a support about which it can rotate and at the other to a mass $m$.

\begin{enumerate}
  \item {[9 pts]} First, consider the geometry confined the surface of a frictionless table. Identify the conserved quantities and write them out in terms of the generalized coordinates. Next, write down and solve the Lagrange equations of motion assuming an initial position $\mathbf r_{0} = (\ell + x_{0}, 0)$ and velocity $\mathbf v_{0} = (0, v_{0})$. (Assume the mass always stays far from the support.)
  \item {[9 pts]} Next, consider the geometry where the spring is hanging vertically from the support as if it were a pendulum. Identify the conserved quantities and write them out in terms of the generalized coordinates. Next, write down the Lagrange equations of motion and describe the motion about the equilibrium point.
  \item {[4 pts]} Describe in words any similarities or differences in conserved quantities in these two physical scenarios.
\end{enumerate}
\begin{center}
  \includegraphics{spring_pivot_hw2.pdf}
\end{center}

\problem{[16 pts] Spring and incline}

A mass $m$ is attached to a spring of spring constant $k$ that can slide vertically on a pole without friction, and moves along a frictionless inclined plane as shown in the figure. After initial displacement along the plane, the mass is released. Derive Euler-Lagrangian equation of motions, and find an expression for the $x$ and $y$ position of the mass as a function of time. The initial displacement of the mass is $x_{0}$. You may assume that the object never slides down the ramp so far that it strikes the floor.
\begin{center}
\includegraphics{slope_spring_hw2.pdf}
\end{center}

\problem{[22 pts] Bead on a parabolic wire}

A bead of mass $m$ slides along a smooth wire bent into a parabolic shape $z = \frac1{2a}r^{2}$ (vertical direction $z$ and radial direction $r$). The wire pivots about the origin and is spinning around the vertical with angular velocity $\omega$.

\begin{enumerate}
  \item {[10 pts]} Identify suitable generalized coordinates and constraints to describe the bead's motion. Write down the Lagrangian and derive the equations of motion (but do not solve).
  \item {[4 pts]} Identify the frequency $\omega$ at which the bead, initialized with no radial velocity, remains at a constant radial position for all times.
  \item {[8 pts]} Identify any conserved quantities and compute them. Discuss their origins and relation to energy ($\frac12 m v^{2} + m g z$), momentum ($m \mathbf v$), or angular momentum ($m \mathbf v \times \mathbf r$).
\end{enumerate}

\end{document}
